\chapter{Linear System Analysis for Optical Imaging Systems}
\label{chpt:linsys}

\section{1D Linear Systems}

In signal processing, systems are essentially transformations between function spaces, where signals (either discrete or continuous) are elements of these function spaces.
If such a transformation is linear, we call it a linear system.

\begin{defn}[Impulse response]
\label{def:impulse-response}
Given a linear system $\mathcal H$, $\mathcal H(\delta(t)) = h(t)$ is the impulse response of the system.
\end{defn}

\begin{defn}[Convolution]
Given a linear system $\mathcal H$ and an input signal $x(t)$, we can represent its output using a convolution:
\begin{align*}
    \mathcal H(x(t))
    &= \mathcal H \left(\int_{-\infty}^\infty x(\tau) \delta(t - \tau) \d\tau \right) \tag{Sampling property.} \\
    &= \int_{-\infty}^\infty \mathcal H(x(t) \delta(t-\tau)) \d\tau \tag{$\mathcal H$ is linear.} \\
    &=\int_{-\infty}^\infty x(t) \mathcal H(\delta(t-\tau)) \d\tau \tag{$x(t)$ is constant.} \\ 
    &=\int_{-\infty}^\infty x(t) h(t-\tau) \d\tau \tag{Defn \ref{def:impulse-response}.} \\ 
    &= x(t) * h(t) \\
    &\equiv y(t) \,.
\end{align*}
Equivalently, in the frequency domain, we can say 
\begin{equation}
    Y(f) = H(f) X(f) = \mathcal F\{y(t)\} \,.
\end{equation}
\end{defn}

\begin{defn}[LTI Systems]
A linear time-invariant (LTI) system is a particular type of linear system with an additional property.
If $x_1$ and $x_2$ are two related signals such that $x_2(t) = x_1(t-t_0)$, then $y_2(t) = \mathcal H(x_2(t)) = \mathcal H(x_1(t-t_0)) = y_1(t)$.
\end{defn}

If we assume that every component of an imaging system is able to be modeled as an LTI system, we can easily chain components in either domain:
\begin{align*}
    h(t) &= h_1(t) * h_2(t) * \cdots * h_k(t) \\
    H(f) &= H_1(f) H_2(f) \cdots H_k(f)
\end{align*}

\begin{defn}[Cross-correlation]
Cross-correlation is a measure of similarity between two signals $x_1$ and $x_2$:\footnote{This looks very similar to the convolution operator -- in fact, cross-correlation is equivalent to convolution when one of the signals is conjugated and time-reversed.}
\begin{equation}
    (x_1 \star x_2)(t) = \int_{-\infty}^\infty \overline{x_1}(\tau)x_2(t+\tau) \d\tau \,,
\end{equation}
where $\overline{x_1}$ represents the complex conjugate of $x_1$.
\end{defn}

\begin{defn}[Autocorrelation]
Autocorrelation is the cross-correlation of a signal with itself.
\end{defn}

\begin{theorem}[Wiener-Khinchin]
The autocorrelation of a signal is equal to the Fourier transform of the absolute value squared of the Fourier transform of the signal:
\begin{equation}
    \mathcal F\{|X(f)|^2\}(t) = (x \star x)(t) \,.
\end{equation}
\end{theorem}

\section{2D Linear Systems}

In 2D systems, the impulse response is commonly referred to as the \textbf{point-spread function} (PSF) since now the impulse represents a single point in 2D space.

Convolutions also have a higher dimensional definition:
\begin{equation}
    (h * x)(u, v) = \int_{-\infty}^\infty \int_{-\infty}^\infty x(u', v') h(u - u', v - v') \d u' \d v' \,,
\end{equation}
where $u, v$ are now spatial coordinates.
Analogous definitions for cross-correlation and autocorrelatoin are also similarly defined.