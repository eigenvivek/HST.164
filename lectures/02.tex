\chapter{Fourier Fundamentals}

\section{Basics of the Fourier Transform}

\begin{defn}[$K$-space]
$K$-space is a 2D/3D Fourier transform of a measured MR image.
That is, if we have two frequencies in 2D $K$-space, $f_x$ and $f_y$, the magnitude $K(f_x, f_y)$ gives us the coefficient for the basis function $e^{-i 2\pi(f_x x + f_y y)}$.
\end{defn}

\begin{theorem}
For natural images, the largest value in $K$-space always occurs at the origin.
\end{theorem}
\begin{proof}
First, simplify the Fourier transform:
\begin{align*}
    F(\omega)
    &= \int_\R f(t) e^{-i\omega t} \d t \\
    &\leq \left| \int_\R f(t) e^{-i\omega t} \d t \right| \\
    &\leq \int_\R |f(t)| |e^{-i\omega t}| \d t \\
    &= \int_\R |f(t)| \d t \tag{$e^{-i\omega t}$ is the unit circle.} \,.
\end{align*}
Since $f(t)$ represents a natural image, it is always non-negative.
Therefore, for all $\omega$, $F(\omega) \leq \int_\R f(t) \d t = F(0)$.
\end{proof}

Some important Fourier transform relationships:
\begin{itemize}
    \item Rect $\stackrel{\mathcal F}{\longrightarrow}$ sinc
    \item Sinc $\stackrel{\mathcal F}{\longrightarrow}$ rect
    \item Gaussian $\stackrel{\mathcal F}{\longrightarrow}$ Gaussian\footnote{Here, Gaussian refers to a signal of the form $e^{-at^2}$}
\end{itemize}