\chapter{Magnetic Resonance Imaging}

\newthought{This lecture covers} basic properties of magnetic resonance (MR) images and physics-based derivation of the principles of MRI.

\section{Properties of MR Images}

We focus on three key properties of MR images:

\begin{itemize}
    \item \textbf{Spatial Resolution:} What is the dimension of every voxel in the images (units of pixels per unit volume)?
    \item \textbf{Temporal Resolution:} How many samples can we acquire per second?
    \item \textbf{Signal-to-Noise Ratio:} SNR is directly proportional the amount of signal in a voxel. This is inversely proportional to spatial and temporal resolution, and directly proportional to acquisition time.
\end{itemize}

\begin{defn}[SNR]
The signal-to-noise ratio (SNR) is defined as $S / \sigma_\epsilon$, where $S$ is the signal and $\sigma_\epsilon$ is the standard deviation of the image noise.
It can also be defined as
\begin{equation}
    \text{SNR} = \frac{\text{FOV}_x}{N_x} \cdot \frac{\text{FOV}_y}{N_y} \cdot \frac{\text{FOV}_z}{N_z} \sqrt{\frac{N_y N_z N_\text{EX}}{1 / \Delta T}} \,,
\end{equation}
where $\text{FOV}$ is the field-of-view, $N_x$ is the number of pixels in the $x$-direction, $N_\text{EX}$ is the number of excitations, and $\Delta T$ is the time sampling interval.\footnote{Note that $\frac{N_x}{\text{FOV}_x}$ is the $x$-direction spatial resolution, demonstrating that inverse relationship with SNR.}
\end{defn}

\begin{remark}
We assume that both signal and noise variance add linearly, so the SNR for $k$ measurements would be
\begin{equation}
    \text{SNR}_k = \frac{kS}{\sqrt{k\sigma^2_\epsilon}} = \sqrt k ~\text{SNR} \,.
\end{equation}
\end{remark}

\begin{defn}[Volume Averaging]
A phenomena that occurs when voxel size is larger than the object being imaged.\footnote{I.e., poor spatial resolution.}
\end{defn}

\begin{defn}[Impulse Response]
We can model imaging systems as linear time-invariant (LTI) systems. The impulse response\footnote{The response of the system to a delta function input.} is also called the ``point-spread function'' (PSF) in imaging systems.\footnote{See Lecture \ref{chpt:linsys}.}
\end{defn}

\section{Fundamental MR Physics}

\subsection{Nuclear Magnetic Resonance (NMR)}

MR is based on NMR, a fundamental property of certain atoms:

\begin{defn}[Nuclear Magnetic Resonance]
Certain atomic nuclei demonstrate the ability to absorb and re-emit \textbf{radiofrequency energy} when placed in a \textbf{magnetic field}.
\end{defn}

This property is leveraged to create MR images as follows:
\begin{enumerate}
    \item A strong, homogeneous magnetic field $B_0$ is applied, forcing nuclei to align
    \item A radiofrequency (RF) pulse is applied, forcing nuclei out of alignment with $B_0$
    \item The nuclei precess, giving off their own RF signal by the NMR phenomenon (i.e., \textbf{T2 relaxation})
    \item The nuclei eventually realign with $B_0$ (i.e., \textbf{T1 relaxation})
\end{enumerate}
The T1 and T2 relaxation times above are different for different tissues in the body, producing contrast in MR images.\footnote{Derived in \S\ref{sec:t1t2}.}

MR focuses on the Hydrogen (H) atom because of its abundance in the body.
The property of H nuclei (i.e., protons) that we care about is spin:
\begin{defn}[Spin]
The spin of a proton quantifies its angular momentum and magnetic moment.\footnote{This allows the proton to act as a tiny bar magnet.}
\end{defn}
\begin{defn}[Magnetic moment]
The strength and orientation of the magnetic field produced by an object, described as $\mu = \bar h S \gamma$, where $\gamma$ is the gyromagnetic ratio.\footnote{$\gamma$ is specific to the atom.}
\end{defn}
If a spinning proton is placed in a magnetic field, the proton's magnetic moment tries to align the proton with the field. However, because the proton is spinning, it \textit{precesses}\footnote{Like a top not falling to gravity.} about the alignment of the field, producing a variable magnetic field.\footnote{This is exactly the NMR phenomenon.}

\subsection{Measuring NMR}

If you placed a coil of wire near the precessing proton, the changing magnetic field would produce a time-varying magnetic flux $\Phi(t)$ through the coil.
Specifically, $\Phi(t) = -\sin(\omega t)$, where the frequency $\omega$ is exactly the angular momentum of the proton, which is given by the Larmor equation:
\begin{equation}
    \label{eqn:larmor}
    \omega = \gamma B_0\,.
\end{equation}
Additionally, this precession induces a variable electrical current that is measurable by Lenz's law:
\begin{equation}
    \label{eqn:lenz}
    \varepsilon = S(t) = -\frac{\d \Phi}{\d t} \,,
\end{equation}
where $\varepsilon$ is the change in voltage in the wire.
This now gives us a mathematical expression for our measured signal:
% TODO: why the sign change here? convention?
\begin{align}
    S(t)
    &= \frac{\d}{\d t} \sin(\omega t) \tag{By Eqn \ref{eqn:lenz}.} \\
    &= \omega \cos(\omega t) \tag*{} \\
    &= \gamma B_0\cos(\omega t) \tag{By Eqn \ref{eqn:larmor}.} \\
    &= N \sin{(\theta)} \gamma B_0\cos(\omega t) \,. \label{eqn:precess}
\end{align}
In the final line, we have introduced two terms:
$N$, which represents the number of protons contributing the the magnetization\footnote{More on this in \S\ref{sec:boltzmann} }, and $\theta$, which represents the \textit{flip angle} or the angle between the proton's magnetic moment and $B_0$.
The flip angle parameterizes the transition between the two discrete spin states (spin-up and spin-down), and the closer this angle is to 90 degrees, the stronger the signal measured by the coils, $S(t)$.

\subsection{T1 and T2 Relaxation}
\label{sec:t1t2}

While the $N$ protons all start with the same spin angle, they don't spin coherently forever.\footnote{I.e., the signal in Equation \ref{eqn:precess} is dampened as protons gradually get out of phase with one another.}
The rate at which protons go out of phase is given by a decay term in
\begin{equation}
    \label{eqn:FID}
    S(t) = N \sin{(\theta)} \gamma B_0\cos(\omega t) e^{-t / T_2} \,,
\end{equation}
where T2 is an intrinsic property for each tissue.\footnote{As protons gradually go out of phase, their magnetic moments cancel each other out, asymptotically driving $S(t)$ to zero.}
\begin{defn}[Free-induction decay]
Equation \ref{eqn:FID} is referred to as the free-induction decay (FID) signal for a particular tissue.
\end{defn}

T2 relaxation measures the rate of signal decay in the \textit{transverse plane}.\footnote{The plane orthogonal to $B_0$.}
Note that as signal decays in the transverse plane, it grows in the \textit{longitudinal plane}.\footnote{The plane parallel to $B_0$.}
T1 relaxation measures this rate.
To figure out what this signal grows \textit{to}, we need to take a detour into Boltzmann statistics.

\subsection{The Boltzmann Distribution}
\label{sec:boltzmann}

We briefly mentioned that spin-$\frac{1}{2}$ nuclei exist in one of two states: spin-up $| \psi_{\uparrow} \rangle$ and spin-down $| \psi_{\downarrow} \rangle$.
Spin-up nuclei (i.e., those aligned with the field) have slightly lower energy than spin-down nuclei, and thus nuclei are more likely to exist in this state.\footnote{A defining principle of quantum mechanics is that, for $T > 0$ (absolute), the probability of nuclei existing in exactly one state is zero.}
When a pair of nuclei in different spin states interact, their magnetic moments cancel out.
The total magnetization $M_0$ is then given by the number of spin-up nuclei that do not have a spin-down partner.
The proportion of nuclei in each state is given by the Boltzmann (aka Gibbs) distribution:
\begin{defn}[Boltzmann distribution]
For spin states $i \in \{ \uparrow, \downarrow \}$, the probability of being in state $i$ is
\begin{equation}
    \Pr{i} = \frac{e^{- E_i / kT}}{\sum_i e^{- E_i / kT}} \,,
\end{equation}
where $E_i$ is the energy of the $i$-th state, $k$ is the Boltzmann constant, and $T$ is the temperature.\footnote{Note the similarity to the logistic distribution.}
\end{defn}

Defining the polarization of the spins as $P = \Pr{\uparrow} - \Pr{\downarrow}$, we can augment Equation \ref{eqn:precess} to be
\begin{equation}
    S(t) = N P \sin{(\theta)} \gamma B_0\cos(\omega t) \,,
\end{equation}
where here, $N$ represents the total number of protons in the system.

\subsection{Spin Echo and Repetition Times}

Using the definition of the total magnetization $M_0$, we can define the longitudinal and transverse components of the magnetic field strength:
\begin{align}
    M_L(t) &= M_0 (1 - e^{-t / T_1}) \\
    M_T(t) &= M_0 e^{-t / T_2}
\end{align}
After delivering a 90 degree excitation pulse, strength of the transverse signal is measured at some \textit{time of echo}, $TE < T_2$.
The system is then allowed to recover longitudinal magnetization by processing for a \textit{time of repetition}, $TR < T_1$, and the process is repeated over and over.
This yields the ultimate MR contrast equation:
\begin{equation}
    S(TE, TR) = M_0 (1 - e^{- TR / T_1}) e^{- TE / T_2} \,.
\end{equation}
Depending on which time constant is prioritized, we end up with either a T1- or a T2-weighted MRI.